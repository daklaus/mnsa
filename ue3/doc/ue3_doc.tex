\documentclass[paper=a4, fontsize=11pt]{scrartcl}
\usepackage[svgnames]{xcolor}
\usepackage[a4paper,pdftex]{geometry}

\usepackage[german]{babel}
\usepackage[utf8x]{inputenc}
\usepackage[T1]{fontenc}
\usepackage{lmodern}

\usepackage{fullpage}
\usepackage{fancyhdr}

%\usepackage{inconsolata}

\usepackage{color}

\definecolor{pblue}{rgb}{0.13,0.13,1}
\definecolor{pgreen}{rgb}{0,0.5,0}
\definecolor{pred}{rgb}{0.9,0,0}
\definecolor{pgrey}{rgb}{0.46,0.45,0.48}

\usepackage{listings}
\lstset{language=Java,
  showspaces=false,
  showtabs=false,
  breaklines=true,
  showstringspaces=false,
  breakatwhitespace=true,
  commentstyle=\color{pgreen},
  keywordstyle=\color{pblue},
  stringstyle=\color{pred},
  basicstyle=\ttfamily,
  moredelim=[il][\textcolor{pgrey}]{$$},
  moredelim=[is][\textcolor{pgrey}]{\%\%}{\%\%}
}

% ------------------------------------------------------------------------------
% Header and Footer Setup
% ------------------------------------------------------------------------------
\pagestyle{fancy}
	\fancyhead[L]{INSO - Industrial Software\\ \small{Institut für Rechnergestützte Automation | Fakultät für Informatik | Technische Universität Wien}}
	\fancyhead[C]{}
	\fancyhead[R]{}
	\fancyfoot[L]{JavaCard Taschenrechner} % Title
	\fancyfoot[C]{}
	\fancyfoot[R]{Seite \thepage}
	\renewcommand{\headrulewidth}{1pt}
	\renewcommand{\footrulewidth}{1pt}
\setlength{\headheight}{0.5cm}
\setlength{\headsep}{0.75cm}

\usepackage{hyperref}
	\hypersetup{
		colorlinks,
		citecolor=black,
		filecolor=black,
		linkcolor=black,
		urlcolor=black
	}

\usepackage{pdfpages}

\usepackage[babel,german=quotes]{csquotes}

% ------------------------------------------------------------------------------
% Title Setup
% ------------------------------------------------------------------------------
\newcommand{\HRule}[1]{\rule{\linewidth}{#1}}

\makeatletter
\def\printtitle{%
	{\centering \@title\par}}
\makeatother

\makeatletter
\def\printauthor{%
	{\centering \normalsize \@author}}
\makeatother


\title{	\normalsize Erstellung eines JavaCard Taschenrechners% Subtitle of the document
	\\[2.0cm] \HRule{0.5pt} \\
	\LARGE \textbf{\uppercase{Dokumentation Übung 2}} % Title
	\HRule{2pt} \\[1.5cm]
	\normalsize Wien am \today \\
	\normalsize Technische Universität Wien \\[1.0cm]
	\normalsize ausgeführt im Rahmen der Lehrveranstaltung \\
	\LARGE 183.286 – (Mobile) Network Service Applications
}


\author{
	Gruppe 37 \\[2.0cm]
	Klaus Krapfenbauer \\
	0926457 \\
	E 066 937 \\
	Software Engineering and Internet Computing \\[2.0cm]
	Christian Ohrfandl \\
	0926341 \\
	E 066 937 \\
	Software Engineering and Internet Computing \\[2.0cm]
}

\begin{document}
% ------------------------------------------------------------------------------
% Title Page
% ------------------------------------------------------------------------------
\thispagestyle{empty} % Remove page numbering on this page
\printtitle
	\vfill
\printauthor

% ------------------------------------------------------------------------------
% Table of Contents
% ------------------------------------------------------------------------------
\newpage
\tableofcontents

% ------------------------------------------------------------------------------
% Document
% ------------------------------------------------------------------------------
\newpage

\section{Kurzfassung}
Es soll über eine Java Anwendung das Telefon mit AT-Commandos so angesteuert werden, dass SMS versendet werden können.

\section{Einleitung}
Ziel dieser Übung war es, das Modem  des Mobiletelefons über die serielle USB-Schnittstelle mit AT-Commandos zu versorgen. Durch AT-Commandos ist es möglich, Modems zu steueren und zu parametrieren. So werden AT-Commandos unter anderem zum Absetzen von Telefongesprächen oder verschicken von SMS verwendet. Der Versand von SMS unterscheidet sich grundsätzlich durch zwei Modi: Text- und PDU-Modus. Für das Erfüllen des Übungsziels musste der PDU-Modus implementiert werden, auf welchen sich die nachfolgende Ausarbeitung beschränkt. Die benötigten Nachrichten, Telefonnummern und Kommunikationseinstellungen sollen aus einer CSV- und Properties-Datei ausgelesen werden.

\section{Problemstellung / Zielsetzung}
Generell wurde diese Übung wie folgt fertiggestellt:
\begin{enumerate}
	\item Studium einschlägiger Literatur (von LVA Leitung bereitgestellt)
	\item Erstellen der Klassen "`Sms"', "`SmsApp"' "`SmsDataPart"' und "`SmsService"'. Des Weiteren haben wir noch jeweils ein Package für das Auslesen von Properties- und CSV-Dateien erstellt. Da wir in diesem Projekt einige Routinen aus der ersten Übung wiederverwenden konnten, haben wir außerdem noch eine Utility-Klasse namens "`NumberConverter"' implementiert.
	\item Folgende Probleme galt es zu lösen:
		\begin{enumerate}
			\item Absetzen von AT-Commandos über die serielle USB-Schnittstele.
			\item Konvertierung von Textzeichen in das 3GPP TS 23.038 / GSM 03.38 7-Bit Standard-Alphabet (Basic Character Set) mit Erweiterungen (Basic Character Set Extension).
			\item Bitweises Konvertieren der zu sendenden Daten und Informationen in das PDU Format.
		\end{enumerate}
\end{enumerate}

\section{Technischer Aufbau}
Generell haben wir das Projekt mit Hilfe von Maven erstellt. Wie bereits erwähnt, besteht diese Lösung aus vier Haupklassen, welche nachfolgend näher vorgestellt werden.

\subsection{"'Sms"'-Klasse}
Model-Klasse zur Speicherung der benötigten SMS-Daten in Plain-Text (Nachrichteninhalt und Telefonnummer)

\subsection{"'SmsApp"'-Klasse}
Hierin befindet sich die Main-Methode zum Ausführen der gewünschten Funktionalität und Routinen für das Initialisieren des Telefon-Modems (ATZ, PIN, SMSC, ...).

\subsection{"'SmsDataPart"'-Klasse}
Model-Klasse für die Speicherung von Nachrichten im PDU-Format

\subsection{"'SmsService"'-Klasse}
Diese Klasse beherbergt Routinen für die 7-Bit Konvertierung von Nachrichteninhalten und Telefonnummern, die richtige Formatierung von (Multipart)-SMS sowie die Konvertierung in das PDU Format.

\subsection{Bibliotheken}

\subsubsection{OpenCSV}
Wurde für das korrekte Auslesen der sendsms.csv Datei verwendet.

\subsubsection{nrjavaserial}
Diese Bibliothek stellt Routinen für die serielle Kommunikation mit dem Modem des Mobiltelefons zur Verfügung (wurde bereits in der ersten Übung verwendet).

\subsubsection{guava}
Die "`Google Utilities"'-Bibliothek wurde für die Verwendung der "`BiMap"' und deren Implementierung "`ImmutableBiMap"' eingesetzt, welche wir für die Abbildung des 7-Bit Alphabets herangezogen haben.

\section{Programmablauf}
Durch Start der main-Methode wird das Telefon-Modem initialisiert, die gespeicherten Nachrichten und Telefonnumern ausgelesen und der Aufbau der SMS gestartet. Für den SMS-Versand werden Telefonnummern und Nachrichteninhalte in das 7-Bit Format konvertiert und anschließend nach den PDU-Spezifikationen bitweise umgewandelt. Wurden diese Vorgänge durchgeführt, werden die SMS unter Verwendung von AT-Commands erfolgreich über das Telefon-Modem versandt.

\section{Resultat}
Unsere Gruppe konnte alle Anforderungen erfolgreich implementieren (siehe Unit-Tests und main-Methode).

\section{Installation}
Das Projekt wird über Maven verwaltet. Es ist keine weitere Konfiguration nötig. Der Befehl "`mvn test"' erstellt das Projekt und führt die Unit-Tests aus. Um die main-Methode zu testen, muss das Telefon mit dem Computer verbunden und die Datei "`sendsms.properties"' an die jeweilige Umgebung angepasst sein (COM-Port des Telefonmodems).

\section{Erkenntnisse}
Aufgrund der Tatsache, dass wir recht früh damit begonnen haben, uns mit, diese Übung betreffender, Materie auseinanderzusetzen, konnten wir das Projekt recht zügig fertigstellen. Die eigentliche Schwierigkeit dieser Übung lag (wieder) am Konvertieren von Bits/Bytes. Der Sendevorgang von (Multipart-)SMS über AT-Commandos konnte schnell umgesetzt werden. Insgesamt kann gesagt werden, dass dies die spaßigste Übung war. :) Einziges Manko: Die Belastung des "`Geldbörserls"' aufgrund des SMS Versands ;)
\end{document}